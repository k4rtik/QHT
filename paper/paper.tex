\documentclass[acmsmall,nonacm,review,timestamp]{acmart}

\acmJournal{PACMPL}

\authorsaddresses{}

\bibliographystyle{ACM-Reference-Format}
\citestyle{acmauthoryear}   %

\usepackage{booktabs}   %
\usepackage{subcaption} %

\usepackage[braket]{qcircuit}
\usepackage{mathpartir}
\usepackage{xspace}
\usepackage[nowatermark]{fixmetodonotes}

\newcommand{\sqir}{\ensuremath{s\textsc{qir}}\xspace}
\newcommand{\qwire}{\ensuremath{\mathcal{Q}\textsc{wire}}\xspace}
\newcommand{\denote}[1]{\llbracket #1 \rrbracket\xspace}

\newcommand{\Z}{\ensuremath{\mathbf{Z}}\xspace}
\newcommand{\X}{\ensuremath{\mathbf{X}}\xspace}
\newcommand{\E}{\ensuremath{\mathbf{E}}\xspace}
\renewcommand{\G}{\ensuremath{\mathbf{G}}\xspace}

\newcommand\hoare[3]{\{\,#1\,\}&#2&\{\,#3\,\}}

\begin{document}

\title{Quantum Hoare Types}         %

\author{Kartik Singhal}
\orcid{0000-0003-1132-269X}             %
\affiliation{
  \department{Department of Computer Science}              %
  \institution{University of Chicago}            %
  \country{USA}                    %
}
\email{ks@cs.uchicago.edu}          %


\begin{abstract}
	Quantum Computing is a new and upcoming way to utilize Quantum Mechanics to do computation. It is easy to introduce bugs while programming quantum computers similar to classical computing. Strong static type systems have prevented a lot of bugs from being introduced in the classical paradigm. There is a similar need for well-designed types in the domain of quantum computing. We introduce Quantum Hoare Types that let programmers reason statically about certain semantic properties of their quantum programs and further allow limited form of formal verification.

\end{abstract}

\keywords{Quantum Computing, Programming Languages, Type Systems, Dependent Types}  %


\maketitle


\section{Introduction}
%(1 page)
%
%\textit{- Describe the problem}
%
%\textit{- State your contributions}


It is difficult to reason about the correctness of quantum programs; both while running them and while writing them. Sound static type systems help prevent a huge class of bugs from occurring, but since the realm of quantum programming is still new, there is not a lot of consensus on what kind of types make the most sense and whether they help programmers reason about the semantic properties associated with the quantum algorithms that they are implementing.

Previous work such as Proto-Quipper\cite{ross2015algebraic} and \qwire\cite{paykin2018,rand2018a} utilize a linear type system and dependent types to enforce a small subset of semantic properties in the quantum setting such as the no-cloning theorem and whether a unitary operator is of the right dimension. These advances in quantum type systems although helpful, still fall short in encoding and enforcing even more useful properties that one would like to be able to express for the purpose of verification. Inspired by the use of Hoare triples in the verification of imperative programs and building on the idea of Hoare Types in classical programming languages\cite{nanevski2008}, we propose Quantum Hoare Types aimed at enabling both sound static type checking and formal verification of quantum programs.

%\subsection{Contributions}
%(bulleted list of contributions)
%
%\textit{In this section we present the main contributions of the paper\ldots}

\section{Problem}
%(1 page)
%
%\textit{Use an example to introduce the problem}

\subsection{Motivating Example}

\begin{figure}[hb]
	\centerline{
		\Qcircuit @C=1em @R=.7em {
		\lstick{\ket{q_0}} & \gate{H} & \ctrl{1} & \qw \\
		\lstick{\ket{q_1}} & \qw      & \targ    & \qw
	}}
	\caption{Bell Pair Creation}
	\label{fig:bellpair}
\end{figure}

In Figure \ref{fig:bellpair}, we see a standard Bell Pair generation circuit.

Assume the input is \ket{00}. A Hadamard gate is applied to qubit $q_0$ and then a CNOT gate is applied to its outcome as the control and the qubit $q_1$ as the target.

The result will be a 2-qubit entangled state aka the first Bell state that will, on measurement, collapse into either \ket{00} or \ket{11} with equal probability.


\subsection{Desired Properties}
For our example, we would like to guarantee correctness of properties such as the following:
\begin{itemize}
	\item Before application of H gate, the qubit $q_0$ should have a deterministic integer value (on measurement)
	\item After application of H, $q_0$ should be in 50-50 probabilistic distribution of being in the two computational states
	\item After application of CNOT, the two qubits should be in a maximally entangled state
	\item Given a certain input to the whole circuit, we should get the corresponding Bell state
\end{itemize}

\section{Idea}
%(2 pages)
%
%\textit{The main idea of this paper is \ldots}

Recent work at Princeton as part of EPiQC\footnote{EPiQC: Enabling Practical-Scale Quantum Computation:  \url{https://epiqc.cs.uchicago.edu}}\cite{huang2018,huang2019} has identified several classes of bugs in quantum programs and proposed approaches to tackle them. They proposed assertion checking using preconditions and post-conditions as one of the most effective techniques for most such classes of bugs.

As programming languages researchers, we think it will be more useful to encode such assertions into a static type system for both formal verification and aiding the programmers in writing correct programs from the start.

\section{Details}
%(5 pages)

\subsection{Abstract Syntax}

\TODO{Figure out if stabilizer types fit well as assertions (need to frame them as containment relation, at the least)}

See Fig. \ref{fig:syntax}.

\begin{figure}[hb]
\begin{tabular}{lrcl}
	\textit{Types} & $\tau, \sigma$ & ::= & $ \mathrm{qbit} \mid \tau \otimes \sigma \mid 1 \mid \Pi x{:}\tau.\sigma \mid \Psi. X. \{P\} x{:}\tau \{Q\}$ \\
	\textit{Assertions} & $P, Q, R$ & ::= & \begin{minipage}[t]{0.6\columnwidth}%
		$ \mathrm{id}_\tau(M,N) \mid \mathrm{seleq}_\tau(H,M,N) \mid \top \mid \bot \mid P \wedge Q \mid P \vee Q \mid \\
		P \supset Q \mid \neg P \mid \forall h{:}\mathrm{heap}.P \mid \Z \mid \X \mid \E \mid \G $
	\end{minipage}\\
	\textit{(Quantum) Heaps} & $H, G$ & ::= & $ h \mid \mathrm{empty} \mid \mathrm{upd}_{\tau}(H, M, N) $\\
	\textit{Elim terms} & $K, L$ & ::= & $ x \mid K\ M \mid M : \tau $\\
	\textit{Intro terms} & $M, N, O$ & ::= & \begin{minipage}[t]{0.6\columnwidth}%
		$ K \mid (\,) \mid \lambda x.\ M \mid \mathrm{do}\ E \mid \ket{0} \mid \ket{1} \mid (M, N) \mid M==N $
	\end{minipage}\\
	\textit{Commands} & $c$ & ::= & \begin{minipage}[t]{0.6\columnwidth}%
		$ \mathrm{alloc}_\tau\ M \mid \mathrm{dealloc}\ M \mid \mathrm{if}_\tau\ M\ \mathrm{then}\ E_1\ \mathrm{else}\ E_2 \mid
		\mathrm{fix}\ f(y{:}\tau){:}\sigma = \mathrm{do}\ E\ \mathrm{in\ eval}\ f M \mid \mathrm{apply}(M, N) \mid \mathrm{capply}(M, N, O) $
	\end{minipage}\\
	\textit{Computations} & $E, F$ & ::= & $ \mathrm{return}\ M \mid x \gets K; E \mid x \Leftarrow c; E \mid x =_\tau M; E $\\
	\textit{Variable context} & $\Delta, \Psi$ & ::= & $ \cdot \mid \Delta, x{:}\tau $\\
	\textit{Heap context} & $X$ & ::= & $ \cdot \mid X, h $\\
	\textit{Assertion context} & $\Gamma$& ::= & $\cdot \mid \Gamma, P $
\end{tabular}
	\caption{Syntax}
\label{fig:syntax}
\end{figure}


\textit{Terms.} Split into introduction and elimination terms. Intro term for Hoare type is do $E$ that represents effectful computation and will suspend the evaluation; corresponding elimination form (which is a computation) is $x \gets K; E$.

\textit{Commands.} \textit{apply} is for unitary circuit application (apply circuit M to N), \textit{capply} that with controls M, circuit N and arguments O. (An alternative is to formulate gate/circuit application as assignment.)

\textit{Computations.} All effectful stuff happens here. Separated by semicolon. Should terminate with a return value. return $M$ and $x \gets K; E$ correspond to monadic \textit{unit} and \textit{bind}. $x =_\tau M$ is syntactic sugar for the let-binding of $M:\tau$ to $x$.

Simple example to create a Bell Pair using QHTT commands:

\begin{center}
\begingroup
\setlength{\tabcolsep}{0pt}
\begin{tabular}{ll}

	$\mathrm{plus\_minus} =_{\mathrm{qbit} \to \mathrm{qbit}} \lambda q.\ \mathrm{do}\ ($&$a \Leftarrow \mathrm{apply}(\mathrm{Hadamard}, q);$\\
																																	&$\mathrm{return}\ a);$\\
\end{tabular}

\begin{tabular}{ll}
	$\mathrm{share} =_{\mathrm{qbit} \to \mathrm{qbit} \otimes \mathrm{qbit}} \lambda a.\ \mathrm{do}\ ($&$q \Leftarrow \mathrm{alloc_{qbit}}\ \ket{0};$\\
		&$b \Leftarrow \mathrm{capply}(\mathrm{a, QNot}, q);$\\
		&$\mathrm{return}\ (a, b));$\\
	$a \gets \mathrm{plus\_minus} \ket{0};$ & \\
	$\mathrm{bell_{00}} \gets \mathrm{share}\ a;$ & \\
	$\mathrm{return}\ \mathrm{bell_{00}}$&
\end{tabular}
\endgroup
\end{center}

In above, we can abbreviate the last two lines to: eval(share $a$).

\textit{Types.} Primitive types such as qbit, pair, unit, dependent functions, and the Hoare type.

\textit{Assertions.} We introduce the notion of stabilizer types \Z, \X, \E, \G as type assertions. \TODO{shouldn't these be really treated as type refinements?}

\subsubsection{Example}
\begin{enumerate}
	\item Bell Pair:
\[ \begin{array}{rcl}
\hoare{a, b : \Z}{H(a);}{a : \X; b: \Z} \\
\hoare{a : \X; b: \Z}{CX(a, b);}{(a, b): \E} \\
\end{array} \]

	\item Superdense coding:
\[ \begin{array}{rcl}
\hoare{x, y : \Z; (a, b): \E}{CZ(x,a); \; CX(y, a); \; transmit(A, B);}{x, y : \Z; (a, b): \E} \\
\hoare{x, y : \Z; (a, b): \E}{CX(a, b);}{x, y : \Z; (a, b): \X \ast \Z} \\
\hoare{x, y, b : \Z; a: \X}{H(a);}{x, y, a, b : \Z} \\
\end{array} \]

or for the complete program:

\[ \begin{array}{rcl}
\hoare{x, y : \Z; (a, b): \E}{ superdense(x, y, a, b);}{x, y, a, b : \Z} \\
\end{array} \]

\TODO{figure out ownership types and pure vs mixed state types}

\newpage

	\item Quantum Teleportation

	We have three qubits. Qubit 0 represents the message in state $\ket{\psi} = \alpha \ket{0} + \beta \ket{1}$. Qubit 1 corresponds to Alice and qubit 2 corresponds to Bob and they both start in the state $\ket{0}$. To send the message, first step in this protocol is for qubits 1 and 2 to be entangled.
\[ \begin{array}{rcl}
\hoare{q_0: qbit = \ket{\psi}; q_1, q_2 : \Z}{H(q_1)}{q_0: qbit; q_2 : \Z; q_1 : \X} \\
\hoare{q_0: qbit = \ket{\psi}; q_2 : \Z; q_1 : \X}{CX(q_1, q_2)}{q_0: qbit; (q_1,q_2) : \E} \\
\end{array} \]

Once entangled they are in the state $\ket{\phi^+} = \frac{\ket{0} + \ket{1}}{2}$. The complete quantum state looks like:

\[  \ket{\psi}\ket{\phi^+} = (\alpha\ket{0} + \beta\ket{1})(\frac{1}{\sqrt{2}}(\ket{00} + \ket{11})) \]

or, normalized:

\[  \ket{\psi}\ket{\phi^+} = \frac{\alpha}{\sqrt{2}}\ket{000} + \frac{\alpha}{\sqrt{2}}\ket{011} + \frac{\beta}{\sqrt{2}}\ket{100} + \frac{\beta}{\sqrt{2}}\ket{111} \]

In the second step we apply CNOT to qubits 0 and 1 and then a Hadamard to qubit 0.

\[ \begin{array}{rcl}
\hoare{q_0: qbit = \ket{\psi}; (q_1,q_2) : \E}{CX(q_0, q_1)}{??} \\
\hoare{??}{H(q_0)}{??} \\
\end{array} \]

After the CNOT application, the resulting state is:

\[ \frac{\alpha}{\sqrt{2}}\ket{000} + \frac{\alpha}{\sqrt{2}}\ket{011} + \frac{\beta}{\sqrt{2}}\ket{110} + \frac{\beta}{\sqrt{2}}\ket{101} \]

and after Hadamard application to qubit 0, it becomes:

\[ \frac{\alpha}{\sqrt{2}}(\frac{1}{\sqrt{2}}(\ket{0} + \ket{1}))\ket{00} + \frac{\alpha}{\sqrt{2}}(\frac{1}{\sqrt{2}}(\ket{0} + \ket{1}))\ket{11} + \frac{\beta}{\sqrt{2}}(\frac{1}{\sqrt{2}}(\ket{0} - \ket{1}))\ket{10} + \frac{\beta}{\sqrt{2}}(\frac{1}{\sqrt{2}}(\ket{0} - \ket{1}))\ket{01} \]

This can be simplified to:

\[ \frac{1}{2}\big[\alpha\ket{00\;0} + \beta\ket{00\;1} + \alpha\ket{01\;1} + \beta\ket{01\;0} + \alpha\ket{10\;0} - \beta\ket{10\;1} + \alpha\ket{11\;1} - \beta\ket{11\;0}\big] \]

The next step is the measurement of the first two qubits:

\[ \begin{array}{rcl}
\hoare{??}{meas(q_0); \; meas(q_1);}{q_0, q_1 : bit, q_2: qbit}\\
\end{array} \]

This leads to 4 possibilities of getting $(0, 0), (0, 1), (1, 0, (1, 1)$. Depending on what we got, we need to make a correction on the resulting state of qubit 2, so we get the original message $\ket{\psi}$. This is done by applying X and Z gates conditioned on the two measured bits.

\[ \begin{array}{rcl}
\hoare{q_0, q_1 : bit, q_2: qbit}{CX(q_1, q_2); \; CZ(q_0, q_2);}{q_0, q_1 : bit, q_2: qbit = \ket{\psi}}\\
\end{array} \]

\TODO{Need a way to assert that $q_2$ has the same quantum state at the end of the program as $q_0$ did at the beginning}

	\item Entanglement Swapping (variant of teleportation):\\
	Initially Alice ($a_1$) and Carol ($c$) share an EPR pair and so do Alice ($a_2$) and Bob ($b$)
\[ \begin{array}{rcl}
\hoare{(c, a_1), (a_2, b): \E}{CX(a_1, a_2);}{??} \\
\hoare{??}{H(a_1)}{??}\\
\hoare{??}{meas(a_1); \; meas(a_2);}{(c,b): \E; a_1, a_2 : bit}\\
\end{array} \]

\item BB84 Protocol for key distribution:
\end{enumerate}

\TODO{Can we assign such type refinements to all/most quantum state patterns that have recognizable structure for human programmers?}

\subsection{Typing Rules}

\subsection{Annotated Example}

\subsection{Evaluation Rules}

\subsection{Notation and Background}
\begin{itemize}
	\item \textbf{Dependent Types}
	\[ \mathrm{inner\_product}: \Pi m: \mathrm{nat}.\mathrm{vector}(m) \times \mathrm{vector}(m) \rightarrow \mathrm{nat} \]

	\item \textbf{Hoare Triples} from \textbf{Floyd-Hoare Logic}
	\[ \{P\}\ C\ \{Q\} \]
	$P$ and $Q$ are pre and postcondition respectively on the program state, while $C$ is the command. For example:
	\[ \{\mathrm{True}\}\ x := 3\ \{x=3\} \]
	\item \textbf{Hoare Types} from \textbf{Hoare Type Theory} (\textbf{HTT}) that combine the strengths of dependent types and Hoare logic.
	\[ \{P\}\ \mathrm{res}:A\ \{Q\}\]
	for a stateful computation executed in a heap that satisfies precondition $P$ and returns a value of type $A$ in a heap that satisfies postcondition $Q$. For example:
	\[ \mathrm{alloc} : \forall \alpha . \Pi x : \alpha. \{\mathrm{emp}\}\ y : \mathrm{nat}\ \{ y \mapsto_{\alpha} x\}\]
\end{itemize}

\subsection{Quantum Hoare Types}
Quantum Hoare typing for the Hadamard and the CNOT operations will look something like:

\begin{mathpar}
	\mathbf{H}(q_0) : \{ \mathrm{q\_state}(q_0) \in \{\ket{0}, \ket{1} \}\}\ r : \mathrm{unit}\ \{ \mathrm{q\_state}(q_0) \in \{\ket{+}, \ket{-}\} \}

	\\

	\mathbf{CNOT}(q_0, q_1) : \{ \mathrm{q\_state}(q_0) \in \{ \ket{+}, \ket{-} \} \wedge \mathrm{q\_state}(q_0) \in \{\ket{0}, \ket{1} \}\ r : \mathrm{unit}\ \{ \mathrm{entangled}(q_0, q_1) \}
\end{mathpar}



Using sequential composition, type of the complete program will be:

\[ \mathbf{BellPair}(q_0, q_1) : \{ \mathrm{q\_state}(q_i) \in \{\ket{0}, \ket{1} \}\}\ r : \mathrm{unit}\ \{ \mathrm{entangled}(q_0, q_1) \} \]

where the predicates need to be encoded in terms of suitable dependent types. This is ongoing work.

\subsubsection{Challenges}
\begin{itemize}
	\item The type system for Hoare Type Theory is fairly complex. We are mechanizing the type system to handle design changes easily.
	\item We need to be able to reason about both decomposable quantum state and entangled state. This aspect may require incorporating ideas from Separation Logic.
\end{itemize}

\subsubsection{Benefits}
\begin{itemize}
	\item Potential to be a unified system for programming, specification and reasoning about quantum programs similar to HTT.
	\item Allows composition of verification in the same style as Hoare logic.
\end{itemize}

\section{Related Work}
%(1-2 pages)

\section{Conclusions and Further Work}

\listofnotes

%(0.5 page)

\begin{acks}                            %
  This material is based upon work supported by
  EPiQC, an \grantsponsor{GS100000001}{NSF}{http://dx.doi.org/10.13039/100000001}
  Expedition in Computing, under Grant
  No.~\grantnum{GS100000001}{1730449}.  Any opinions, findings, and
  conclusions or recommendations expressed in this material are those
  of the authors and do not necessarily reflect the views of the
  National Science Foundation.
\end{acks}

\bibliography{bibfile}

%\appendix
%\section{Appendix}

%Text of appendix \ldots

\end{document}
