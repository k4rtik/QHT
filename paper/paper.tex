%% For double-blind review submission, w/o CCS and ACM Reference (max submission space)
\documentclass[acmsmall,review]{acmart}\settopmatter{printfolios=true,printccs=false,printacmref=false}
%% For double-blind review submission, w/ CCS and ACM Reference
%\documentclass[acmsmall,review,anonymous]{acmart}\settopmatter{printfolios=true}
%% For single-blind review submission, w/o CCS and ACM Reference (max submission space)
%\documentclass[acmsmall,review]{acmart}\settopmatter{printfolios=true,printccs=false,printacmref=false}
%% For single-blind review submission, w/ CCS and ACM Reference
%\documentclass[acmsmall,review]{acmart}\settopmatter{printfolios=true}
%% For final camera-ready submission, w/ required CCS and ACM Reference
%\documentclass[acmsmall]{acmart}\settopmatter{}


%% Journal information
%% Supplied to authors by publisher for camera-ready submission;
%% use defaults for review submission.
\acmJournal{PACMPL}
\acmVolume{1}
\acmNumber{CONF} % CONF = POPL or ICFP or OOPSLA
\acmArticle{1}
\acmYear{2018}
\acmMonth{1}
\acmDOI{} % \acmDOI{10.1145/nnnnnnn.nnnnnnn}
\startPage{1}

%% Copyright information
%% Supplied to authors (based on authors' rights management selection;
%% see authors.acm.org) by publisher for camera-ready submission;
%% use 'none' for review submission.
\setcopyright{none}
%\setcopyright{acmcopyright}
%\setcopyright{acmlicensed}
%\setcopyright{rightsretained}
%\copyrightyear{2018}           %% If different from \acmYear

%% Bibliography style
\bibliographystyle{ACM-Reference-Format}
%% Citation style
%% Note: author/year citations are required for papers published as an
%% issue of PACMPL.
\citestyle{acmauthoryear}   %% For author/year citations


%%%%%%%%%%%%%%%%%%%%%%%%%%%%%%%%%%%%%%%%%%%%%%%%%%%%%%%%%%%%%%%%%%%%%%
%% Note: Authors migrating a paper from PACMPL format to traditional
%% SIGPLAN proceedings format must update the '\documentclass' and
%% topmatter commands above; see 'acmart-sigplanproc-template.tex'.
%%%%%%%%%%%%%%%%%%%%%%%%%%%%%%%%%%%%%%%%%%%%%%%%%%%%%%%%%%%%%%%%%%%%%%


%% Some recommended packages.
\usepackage{booktabs}   %% For formal tables:
                        %% http://ctan.org/pkg/booktabs
\usepackage{subcaption} %% For complex figures with subfigures/subcaptions
                        %% http://ctan.org/pkg/subcaption

%% Added by kartik
\usepackage[braket]{qcircuit}
\usepackage{mathpartir}

\begin{document}

%% Title information
\title[]{Quantum Hoare Types}         %% [Short Title] is optional;
                                        %% when present, will be used in
                                        %% header instead of Full Title.
%\titlenote{with title note}             %% \titlenote is optional;
                                        %% can be repeated if necessary;
                                        %% contents suppressed with 'anonymous'
%\subtitle{Subtitle}                     %% \subtitle is optional
%\subtitlenote{with subtitle note}       %% \subtitlenote is optional;
                                        %% can be repeated if necessary;
                                        %% contents suppressed with 'anonymous'


%% Author information
%% Contents and number of authors suppressed with 'anonymous'.
%% Each author should be introduced by \author, followed by
%% \authornote (optional), \orcid (optional), \affiliation, and
%% \email.
%% An author may have multiple affiliations and/or emails; repeat the
%% appropriate command.
%% Many elements are not rendered, but should be provided for metadata
%% extraction tools.

%% Author with single affiliation.
\author{Kartik Singhal}
%\authornote{with author1 note}          %% \authornote is optional;
                                        %% can be repeated if necessary
\orcid{0000-0003-1132-269X}             %% \orcid is optional
\affiliation{
%  \position{Position1}
  \department{Department of Computer Science}              %% \department is recommended
  \institution{University of Chicago}            %% \institution is required
%  \streetaddress{Street1 Address1}
%  \city{City1}
%  \state{State1}
%  \postcode{Post-Code1}
  \country{United States}                    %% \country is recommended
}
\email{ks@cs.uchicago.edu}          %% \email is recommended

%% Author with two affiliations and emails.
%\author{John Reppy}
%\authornote{with author2 note}          %% \authornote is optional;
                                        %% can be repeated if necessary
%\orcid{nnnn-nnnn-nnnn-nnnn}             %% \orcid is optional
%\affiliation{
%%  \position{Position2a}
%  \department{Department of Computer Science}             %% \department is recommended
%  \institution{University of Chicago}           %% \institution is required
%%  \streetaddress{Street2a Address2a}
%%  \city{City2a}
%%  \state{State2a}
%%  \postcode{Post-Code2a}
%  \country{United States}                   %% \country is recommended
%}
%\email{jhr@cs.uchicago.edu}         %% \email is recommended
%\affiliation{
%  \position{Position2b}
%  \department{Department2b}             %% \department is recommended
%  \institution{Institution2b}           %% \institution is required
%  \streetaddress{Street3b Address2b}
%  \city{City2b}
%  \state{State2b}
%  \postcode{Post-Code2b}
%  \country{Country2b}                   %% \country is recommended
%}
%\email{first2.last2@inst2b.org}         %% \email is recommended


%% Abstract
%% Note: \begin{abstract}...\end{abstract} environment must come
%% before \maketitle command
\begin{abstract}
(4 sentences)

\textit{Text of abstract \ldots.}

\end{abstract}


%% 2012 ACM Computing Classification System (CSS) concepts
%% Generate at 'http://dl.acm.org/ccs/ccs.cfm'.
\begin{CCSXML}
<ccs2012>
<concept>
<concept_id>10011007.10011006.10011008</concept_id>
<concept_desc>Software and its engineering~General programming languages</concept_desc>
<concept_significance>500</concept_significance>
</concept>
</ccs2012>
\end{CCSXML}

\ccsdesc[500]{Software and its engineering~General programming languages}
\ccsdesc[300]{Social and professional topics~History of programming languages}
%% End of generated code


%% Keywords
%% comma separated list
\keywords{keyword1, keyword2, keyword3}  %% \keywords are mandatory in final camera-ready submission


%% \maketitle
%% Note: \maketitle command must come after title commands, author
%% commands, abstract environment, Computing Classification System
%% environment and commands, and keywords command.
\maketitle


\section{Introduction}
(1 page)

\textit{- Describe the problem}

\textit{- State your contributions}


It is difficult to reason about the correctness of quantum programs; both while running them and while writing them. Sound static type systems help prevent a huge class of bugs from occurring, but since the realm of quantum programming is still new, there is not a lot of consensus on what kind of types make the most sense and whether they help programmers reason about the semantic properties associated with the quantum algorithms that they are implementing.

Previous work such as Proto-Quipper\cite{2015arXiv151002198R} and QWire\cite{rand18,paykin18} utilize a linear type system and dependent types to enforce a small subset of semantic properties in the quantum setting such as the no-cloning theorem and whether a unitary operator is of the right dimension. These advances in quantum type systems although helpful, still fall short in encoding and enforcing even more useful properties that one would like to be able to express for the purpose of verification. Inspired by the use of Hoare triples in the verification of imperative programs and building on the idea of Hoare Types in classical programming languages, we propose Quantum Hoare Types aimed at enabling both sound static type checking and formal verification of quantum programs.

\subsection{Contributions}
(bulleted list of contributions)

\textit{In this section we present the main contributions of the paper\ldots}

\section{Problem}
(1 page)

\textit{Use an example to introduce the problem}

\subsection{Motivating Example}

\begin{figure}
	\centerline{
		\Qcircuit @C=1em @R=.7em {
		\lstick{\ket{q_0}} & \gate{H} & \ctrl{1} & \qw \\
		\lstick{\ket{q_1}} & \qw      & \targ    & \qw
	}}
	\caption{Bell Pair Creation}
	\label{fig:bellpair}
\end{figure}

In Figure \ref{fig:bellpair}, we see a standard Bell Pair generation circuit.

Assume the input is \ket{00}. A Hadamard gate is applied to qubit $q_0$ and then a CNOT gate is applied to its outcome as the control and the qubit $q_1$ as the target.

The result will be a 2-qubit entangled state aka the first Bell state that will, on measurement, collapse into either \ket{00} or \ket{11} with equal probability.


\subsection{Desired Properties}
For our example, we would like to guarantee correctness of properties such as the following:
\begin{itemize}
	\item Before application of H gate, the qubit $q_0$ should have a deterministic integer value (on measurement)
	\item After application of H, $q_0$ should be in 50-50 probabilistic distribution of being in the two computational states
	\item After application of CNOT, the two qubits should be in a maximally entangled state
	\item Given a certain input to the whole circuit, we should get the corresponding Bell state
\end{itemize}

\section{Idea}
(2 pages)

\textit{The main idea of this paper is \ldots}

Recent work at Princeton as part of EPiQC\footnote{EPiQC: Enabling Practical-Scale Quantum Computation:  \url{https://epiqc.cs.uchicago.edu}}\cite{DBLP:conf/oopsla/HuangM18,Huang:2019:SAV:3307650.3322213} has identified several classes of bugs in quantum programs and proposed approaches to tackle them. They proposed assertion checking using preconditions and post-conditions as one of the most effective techniques for most such classes of bugs.

As programming languages researchers, we think it will be more useful to encode such assertions into a static type system for both formal verification and aiding the programmers in writing correct programs from the start.

\section{Details}
(5 pages)

\subsection{Notation and Background}
\begin{itemize}
	\item \textbf{Dependent Types}
	\[ \mathrm{inner\_product}: \Pi m: \mathrm{nat}.\mathrm{vector}(m) \times \mathrm{vector}(m) \rightarrow \mathrm{nat} \]

	\item \textbf{Hoare Triples} from \textbf{Floyd-Hoare Logic}
	\[ \{P\}\ C\ \{Q\} \]
	$P$ and $Q$ are pre and postcondition respectively on the program state, while $C$ is the command. For example:
	\[ \{\mathrm{True}\}\ x := 3\ \{x=3\} \]
	\item \textbf{Hoare Types} from \textbf{Hoare Type Theory} (\textbf{HTT}) that combine the strengths of dependent types and Hoare logic.
	\[ \{P\}\ \mathrm{res}:A\ \{Q\}\]
	for a stateful computation executed in a heap that satisfies precondition $P$ and returns a value of type $A$ in a heap that satisfies postcondition $Q$. For example:
	\[ \mathrm{alloc} : \forall \alpha . \Pi x : \alpha. \{\mathrm{emp}\}\ y : \mathrm{nat}\ \{ y \mapsto_{\alpha} x\}\]
\end{itemize}

\subsection{Quantum Hoare Types}
Quantum Hoare typing for the Hadamard and the CNOT operations will look something like:

\begin{mathpar}
	\mathbf{H}(q_0) : \{ \mathrm{q\_state}(q_0) \in \{\ket{0}, \ket{1} \}\}\ r : \mathrm{unit}\ \{ \mathrm{q\_state}(q_0) \in \{\ket{+}, \ket{-}\} \}

	\\

	\mathbf{CNOT}(q_0, q_1) : \{ \mathrm{q\_state}(q_0) \in \{ \ket{+}, \ket{-} \} \wedge \mathrm{q\_state}(q_0) \in \{\ket{0}, \ket{1} \}\ r : \mathrm{unit}\ \{ \mathrm{entangled}(q_0, q_1) \}
\end{mathpar}



Using sequential composition, type of the complete program will be:

\[ \mathbf{BellPair}(q_0, q_1) : \{ \mathrm{q\_state}(q_i) \in \{\ket{0}, \ket{1} \}\}\ r : \mathrm{unit}\ \{ \mathrm{entangled}(q_0, q_1) \} \]

where the predicates need to be encoded in terms of suitable dependent types. This is ongoing work.

\subsubsection{Challenges}
\begin{itemize}
	\item The type system for Hoare Type Theory is fairly complex. We are mechanizing the type system to handle design changes easily.
	\item We need to be able to reason about both decomposable quantum state and entangled state. This aspect may require incorporating ideas from Separation Logic.
\end{itemize}

\subsubsection{Benefits}
\begin{itemize}
	\item Potential to be a unified system for programming, specification and reasoning about quantum programs similar to HTT.
	\item Allows composition of verification in the same style as Hoare logic.
\end{itemize}


\hrulefill

From Xavier Leroy's OPLSS Slides:


Arithmetic expressions:

\[a ::= n \mid x \mid a1 + a2 \mid a1 - a2 \mid a1 \times a2\]

Boolean expressions:

\begin{align*}
	b ::=&\ \mathrm{true} \mid \mathrm{false} \mid a1 = a2 \mid a1 \le a2\\
	\mid&\ \mathrm{not}\ b \mid b1\ \mathrm{and}\ b2
\end{align*}

Commands (statements):

\begin{align*}
	c ::=&\ \mathrm{SKIP}	&\mathrm{(do\ nothing)}\\
	\mid&\ x ::= a	&\mathrm{(assignment)}\\
	\mid&\ 1; ; c2	&\mathrm{(sequence)}\\
	\mid&\ \mathrm{IFB}\ b\ \mathrm{THEN}\ c1\ \mathrm{ELSE}\ c2\ \mathrm{FI}	&\mathrm{(conditional)}\\
	\mid&\ \mathrm{WHILE}\ b\ \mathrm{DO}\ c\ \mathrm{END}	&\mathrm{(loop)}
\end{align*}

\section{Related Work}
(1-2 pages)

\cite{DBLP:conf/oopsla/HuangM18}

\cite{Huang:2019:SAV:3307650.3322213}

\cite{nanevski_morrisett_birkedal_2008}

\cite{paykin18}

\cite{rand18}

\cite{2015arXiv151002198R}

\section{Conclusions and Further Work}

(0.5 page)

%% Acknowledgments
\begin{acks}                            %% acks environment is optional
                                        %% contents suppressed with 'anonymous'
  %% Commands \grantsponsor{<sponsorID>}{<name>}{<url>} and
  %% \grantnum[<url>]{<sponsorID>}{<number>} should be used to
  %% acknowledge financial support and will be used by metadata
  %% extraction tools.
  This material is based upon work supported by
  EPiQC, an \grantsponsor{GS100000001}{NSF}{http://dx.doi.org/10.13039/100000001}
  Expedition in Computing, under Grant
  No.~\grantnum{GS100000001}{1730449}.  Any opinions, findings, and
  conclusions or recommendations expressed in this material are those
  of the authors and do not necessarily reflect the views of the
  National Science Foundation.
\end{acks}


%% Bibliography
\bibliography{bibfile}


%% Appendix
\appendix
\section{Appendix}

Text of appendix \ldots

\end{document}
